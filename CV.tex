% !TEX TS-program = xelatex
% !TEX encoding = UTF-8 Unicode
% -*- coding: UTF-8; -*-
% vim: set fenc=utf-8

%%%%%%%%%%%%%%%%%%%%%%%%%%%%%%%%%%%%%%%%%%%%%%%%%%%%%%%%%%%%%%%%%
%% CV.tex
%% <https://github.com/zachscrivena/simple-resume-cv>
%% This is free and unencumbered software released into the
%% public domain; see <http://unlicense.org> for details.
%%%%%%%%%%%%%%%%%%%%%%%%%%%%%%%%%%%%%%%%%%%%%%%%%%%%%%%%%%%%%%%%%

% See "README.md" for instructions on compiling this document.

\documentclass[letterpaper,MMMyyyy,nonstopmode]{template}
% Class options:
% a4paper, letterpaper, nonstopmode, draftmode
% MMMyyyy, ddMMMyyyy, MMMMyyyy, ddMMMMyyyy, yyyyMMdd, yyyyMM, yyyy

%%%%%%%%%%%%%%%%%%%%%%%%%%%%%%%%%%%%%%%%%%%%%%%%%%%%%%%%%%%%%%%%%
%% PREAMBLE.
%%%%%%%%%%%%%%%%%%%%%%%%%%%%%%%%%%%%%%%%%%%%%%%%%%%%%%%%%%%%%%%%%

% CV Info (to be customized).
\newcommand{\CVAuthor}{Jaeyoon Song}
\newcommand{\CVTitle}{Jaeyoon Song's Curriculum Vitae}
\newcommand{\CVNote}{CV compiled on {\today}}
\newcommand{\CVWebpage}{https://jaeyoon.io}

% PDF settings and properties.
\hypersetup{
pdftitle={\CVTitle},
pdfauthor={\CVAuthor},
pdfsubject={\CVWebpage},
pdfcreator={XeLaTeX},
pdfproducer={},
pdfkeywords={},
unicode=true,
bookmarks=true,
bookmarksopen=true,
pdfstartview=FitH,
pdfpagelayout=OneColumn,
pdfpagemode=UseOutlines,
hidelinks,
breaklinks}

% Shorthand.
\newcommand{\Code}[1]{\mbox{\textbf{#1}}}
\newcommand{\CodeCommand}[1]{\mbox{\textbf{\textbackslash{#1}}}}

%%%%%%%%%%%%%%%%%%%%%%%%%%%%%%%%%%%%%%%%%%%%%%%%%%%%%%%%%%%%%%%%%
%% ACTUAL DOCUMENT.
%%%%%%%%%%%%%%%%%%%%%%%%%%%%%%%%%%%%%%%%%%%%%%%%%%%%%%%%%%%%%%%%%

\begin{document}

%%%%%%%%%%%%%%%
% TITLE BLOCK %
%%%%%%%%%%%%%%%

\Title{\CVAuthor}

\Gap
\begin{SubTitle}
\href{mailto:song@jaeyoon.io}
{song@jaeyoon.io}
\,\SubBulletSymbol\,
\href{\CVWebpage}
{\url{\CVWebpage}}
\end{SubTitle}

\begin{Body}


%%%%%%%%%%%%%%%
%% EDUCATION %%
%%%%%%%%%%%%%%%

\BigGap
\Section
{Education}
{Education}
{PDF:Education}

\Entry
\href{http://www.snu.ac.kr}
{\textbf{Seoul National University}},
Seoul, Korea, Republic of
\hfill
\DatestampYMD{2016}{03}{01} --
Present

\Gap
\BulletItem
B.A.A.,
\href{http://cba.snu.ac.kr}
{Business Administration}

\Gap
\BulletItem
\href{http://cse.snu.ac.kr}
{Computer Science and Engineering (Minor)}

%%%%%%%%%%%%%%%%%%%%%%%%%
%% RESEARCH EXPERIENCE %%
%%%%%%%%%%%%%%%%%%%%%%%%%
\BigGap
\Section
{Research Experience}
{Research Experience}
{PDF:ResearchExperience}

\Entry
\href{https://www.kixlab.org}
{\textbf{KAIST Interaction Lab}},
Korea Advanced Institute of Science and Technology
\hfill
\DatestampYM{2018}{12} -- 
\DatestampYM{2019}{2},

\Gap
\BulletItem
Undergraduate Research Intern
\hfill
\DatestampYM{2019}{6} --
\DatestampYM{2019}{8}

\Gap
\begin{Detail}
\SubBulletItem
Supervisor:
Prof.~Juho~Kim
\SubBulletItem
Project:
SolutionChat - Real-time Moderator Support for Chat-based Structured Discussion
\SubBulletItem
Project:
SuggestBot - Crowdsourcing Evidence for Debate using Amazon Mechanical Turk
\SubBulletItem
Project:
Credibility Assessment and Critical Thinking through Microtasks while Reading
\end{Detail}

\BigGap
\Entry
\href{http://hcil.snu.ac.kr/}
{\textbf{Human-Computer Interaction Lab}},
Seoul National University
\hfill
\DatestampYMD{2018}{06}{20} --
\DatestampYMD{2018}{08}{31}

\Gap
\BulletItem
Undergraduate Research Intern
\begin{Detail}

\SubBulletItem
Supervisor:
Prof.~Jinwook~Seo
\SubBulletItem
Project:
SoundGlance - Briefing the Glanceable Cues of Web Pages for Screen Reader Users

\end{Detail}

    
%%%%%%%%%%%%%%%
%% INTERESTS %%
%%%%%%%%%%%%%%%
\BigGap
\Section
{Interests}
{Interests}
{PDF:Interests}

\Entry
Crowdsourcing,
Online Discussion,
Computer-Supported Cooperative Work,
Future of Work.

%%%%%%%%%%%%%%%%%%
%% PUBLICATIONS %%
%%%%%%%%%%%%%%%%%%

\BigGap
\Section
{Publications}
{Publications}
{PDF:Publications}

\SubSection
{Journals}
{Journals}
{PDF:Journals}

% Declare a new group to limit the scope of \MaxNumberedItem to this subsection.
\begingroup
\renewcommand{\MaxNumberedItem}{[88]}

\BigGap
\NumberedItem{[1]}
  \href{https://doi.org/10.3390/su10093001}
  \underline{J.~Song} and C.~Kim,
  \textbf{What Is Needed for the Sustainable Success of Open Source Software Projects: Efficiency Analysis of Commit Production Process via Git},
  \textit{Sustainability} (SCIE/SSCI),
  vol.~10,
  no.~9,
  \DatestampYM{2018}{08}.
  \vspace{2mm}\newline
  {\small{
    What is needed for open source software projects to be efficient? Linus' Law celebrates the `many eyeballs' as a key advantage of open source projects. 
    Nevertheless, when it comes to efficiency, `many eyeballs' could be a double-bladed sword. By mining and analyzing the data of 34 projects on GitHub, this paper performed data envelopment analysis (DEA) to investigate the efficiency of open source projects.
  }}

\endgroup

\BigGap
\SubSection
{Posters}
{Posters}
{PDF:Posters}

% Declare a new group to limit the scope of \MaxNumberedItem to this subsection.
\begingroup
\renewcommand{\MaxNumberedItem}{[88]}

\BigGap
\NumberedItem{[2]}
  \href{https://doi.org/10.1145/3290607.3312865}
  \underline{J.~Song}, K.~Choe, J.~Jo, and J.~Seo,
  \textbf{SoundGlance: Briefing the Glanceable Cues of Web Pages for Screen Reader Users,}
  \textit{ACM CHI Conference on Human Factors in Computing Systems (CHI 2019 Late Breaking Work)},
  ACM, New York, NY, USA,
  \DatestampYM{2019}{05}.
  \vspace{2mm}\newline
  {\small{
    Although screen readers can convey the textual information or structural properties of a web page, they cannot deliver its overall impression. Such a limitation hinders blind web users from obtaining an overview of the website, which non-blind people can do in a short time. 
    SoundGlance is a novel application that briefly delivers an auditory summary of web pages. SoundGlance supports the screen reader users by converting the important glanceable cues of the pages into sound.
    To automatically extract the glanceable cues, we trained a convolutional neural network (CNN) with the annotations on the screenshots of 39 web pages.
  }}

\endgroup

\newpage

%%%%%%%%%%%%%%%%%%
%% PUBLICATIONS %%
%%%%%%%%%%%%%%%%%%

\BigGap
\Section
{Research Projects}
{Research Projects}
{PDF:OngoingProjects}

\SubSection
{Papers Under Review}
{Papers Under Review}
{PDF:UnderReview}

% Declare a new group to limit the scope of \MaxNumberedItem to this subsection.
\begingroup
\renewcommand{\MaxNumberedItem}{[88]}

\BigGap
\NumberedItem{[3]}
  S.~Lee, \underline{J.~Song}, S.~Park, J.~Kim, J.~Kim, and E.~Ko,
  \textbf{SolutionChat: Real-time Moderator Support for Chat-based Structured Discussion},
  submitted to \textit{ACM CHI Conference on Human Factors in Computing Systems 2020 (CHI 2020).}
  \vspace{2mm}\newline
  {\small{
    Online chat is an emerging channel for discussing community problems. It is common practice for communities to assign dedicated moderators to maintain a structured discussion and enhance the problem-solving experience. However, due to the synchronous nature of online chat, moderators face a high managerial overhead in tasks like discussion stage management, opinion summarization, and consensus-building support. SolutionChat is a system that assists moderators with facilitating a structured discussion for community problem-solving.
    With SolutionChat, we envision untrained moderators to effectively facilitate chat-based discussions of important community matters.
  }}


\BigGap
\NumberedItem{[4]}
  D.~Shin, \underline{J.~Song}, S.~Song, J.~Park, J.~Lee and S.~Jun,
  \textbf{TalkingBoogie: Collaborative Mobile AAC System for Non-verbal Children with Developmental Disabilities and Their Caregivers},
  submitted to \textit{ACM CHI Conference on Human Factors in Computing Systems 2020 (CHI 2020).}
  \vspace{2mm}\newline
  {\small{
    Augmentative and alternative communication (AAC) technologies are widely used to help non-verbal children enable communication. For AAC-aided communication to be successful, caregivers should support children with consistent intervention strategies in various settings. TalkingBoogie supports caregivers to effectively collaborate with one another and create a shared understanding of intervention strategies.
  }}

\endgroup

\BigGap
\SubSection
{Work in Progress}
{Work in Progress}
{PDF:WorkInProgress}

\begingroup
\renewcommand{\MaxNumberedItem}{[88]}

\BigGap
\NumberedItem{[5]}
  \textbf{Credibility Assessment and Critical Thinking through Microtasks while Reading}, advised by Prof. Juho Kim
  \vspace{2mm}\newline
  {\small{
    Fact-checking systems using artificial intelligence (AI) tend to focus on performance, while often neglecting their interactions with people. Although improving the predictive accuracy of ML models is a worthwhile goal, engaging humans into the decision making process is equally important for such approach to be sustainable and scalable. This project aims to design a crowdsourcing system where the readers and AI models can collaboratively identify misinformation in online news articles. To make this happen, we have two research questions in mind: (1) How to co-optimize the maximization of system-side information gain and user-side engagement gain? and (2) How to motivate readers to engage in more tasks? 
  }}

\endgroup


% %%%%%%%%%%%%%%%%%%%%%%%
% %% SUBJECTS %%
% %%%%%%%%%%%%%%%%%%%%%%%
\BigGap
\Section
{Relevant Coursework}
{Relevant Coursework}
{PDF:RelevantCoursework}

\Entry
  \textbf{Seminar in Organizational Behavior},
  Dept. of Business Administration
  \hfill
  \DatestampY{2019} Spring

  \begin{Detail}
  \SubBulletItem
    Graduate-level course that required reading 40 journal articles in total, writing a research proposal every week, and reviewing the proposals of other students during the class discussion. (Final grade: A0)
  \end{Detail}

\BigGap
\Entry
  \textbf{Organizational Psychology},
  Dept. of Psychology
  \hfill
  \DatestampY{2019} Spring

  \begin{Detail}
  \SubBulletItem
    Major theories and issues in the field of organizational psychology. (Final grade: A+)
  \end{Detail}

\BigGap
\Entry
  \textbf{HCI and Communication},
  Dept. of Communication
  \hfill
  \DatestampY{2018} Fall

  \begin{Detail}
  \SubBulletItem
    Topics in robot journalism, human-robot interaction, and social computing. (Final grade: A+)
  \end{Detail}

\BigGap
  \Entry
  \textbf{Human-Computer Interaction},
  Dept. of Computer Science and Engineering,
  \hfill
  \DatestampY{2018} Spring

  \begin{Detail}
  \SubBulletItem
   Introduction to HCI and information visualization. (Final grade: A+)
  \end{Detail}

\newpage

%%%%%%%%%%%%%%%%%%%%%%%%%%%
%%  SCHOLARSHIPS %%
%%%%%%%%%%%%%%%%%%%%%%%%%%%

\Section
{Scholarships}
{Scholarships}
{PDF:Scholarships}

\BigGap
\Entry
\textbf{Yangyoung Foundation Scholarship}
\hfill
\DatestampY{2018} --
Present
\begin{Detail}
\SubBulletItem
Based on both merit and need.
\end{Detail}

\Gap
\Entry
\textbf{Samsung Convergence Software Course Scholarship}
\hfill
\DatestampYM{2018}{06}
\begin{Detail}
\SubBulletItem
Scholarship for successfully finishing the Samsung Convergence Software Course (SCSC) program.
\end{Detail}

\Gap
\Entry
\textbf{Eminence Scholarship (Full)},
Seoul National University
\hfill
\DatestampY{2016} --
\DatestampY{2017}
\begin{Detail}
\SubBulletItem
Merit-based.
\end{Detail}

%%%%%%%%%%%%%%%%%%%%%%%%%%%
%% AWARDS & SCHOLARSHIPS %%
%%%%%%%%%%%%%%%%%%%%%%%%%%%

\Section
{Honors \&\newline Awards}
{Honors \&\newline Awards}
{PDF:Awards}

\BigGap
\Entry
\href{https://sdeprogram.snu.ac.kr}
{\textbf{Undergraduate Research Grant}, Seoul National University
\hfill
\DatestampYMD{2019}{03}{03}
\begin{Detail}
\SubBulletItem
Granted by SNU Undergraduate Research Program (URP).
\SubBulletItem
Topic: Supporting caregivers to collaborate on AAC intervention strategies.
\end{Detail}}

\BigGap
\Entry
\href{https://sdeprogram.snu.ac.kr}
{\textbf{Outstanding Research Award}, Seoul National University
\hfill
\DatestampYMD{2019}{03}{03}
\begin{Detail}
\SubBulletItem
Awarded by Student-Directed Education (SDE) program.
\SubBulletItem
Topic: What is needed for the sustainable success of open source software projects?
\SubBulletItem
Individual work
\end{Detail}}

\Gap
\Entry
\href{https://research.samsung.com/aichallenge/hackathon}
{\textbf{Top Ten Finalist}, Samsung AI Challenge
\hfill
\DatestampYMD{2018}{09}{03}
\begin{Detail}
\SubBulletItem
Awarded by Samsung Research
\SubBulletItem
Topic: Restaurant recommendation service
\SubBulletItem
My role in the team: front-end engineering
\end{Detail}}

\Gap
\Entry
\href{http://www.hangyo.com/news/article.html?no=85459}
{\textbf{Grand Prize}, Undergraduate Research Presentation Competition
\hfill
\DatestampYMD{2018}{05}{10}
\begin{Detail}
\SubBulletItem
Awarded by Korean Production \& Operations Management Society (KOPOMS).
\SubBulletItem
Topic: Two-stage data envelopment analysis (DEA) on open source software projects
\SubBulletItem
Individual work
\end{Detail}}

\Gap
\Entry
\textbf{Top Ten Winner}, Annual Likelion Ideathon
\hfill
\DatestampYMD{2017}{07}{15}
\begin{Detail}
\SubBulletItem
Awarded \$1,000 AWS credits by Likelion and Amazon Korea.
\SubBulletItem
Topic: An idea of `Music of Bullshit,' an online platform where users can collaboratively compose music with any kind of nonsense sound.
\SubBulletItem
Individual work
\end{Detail}

\Gap
\Entry
\textbf{Final Winner}, SNU School Service Development Tournament
\hfill
\DatestampYM{2017}{02}
\begin{Detail}
\SubBulletItem
Awarded by SNULife--an online student community of Seoul National University.
\SubBulletItem
Topic: Shashagungun, a web platform that gathers posters of various school events.
\SubBulletItem
My role in the team: Team leader and front-end engineering
\end{Detail}

%%%%%%%%%%%%%%%%%%%%%%%
%% EXPERIENCE %%
%%%%%%%%%%%%%%%%%%%%%%%

\Section
{Work Experience}
{Work Experience}
{PDF:WorkExperience}    

\BigGap
\Entry
  \href{https://chartmetric.io}
  {\textbf{Chartmetric}, Intern, Front-end Engineering
  \hfill
  \DatestampYM{2017}{08} --
  \DatestampYM{2017}{11}
  }
  \begin{Detail}
  \SubBulletItem
    Chartmetric is a startup based in Palo Alto, providing tools to track, measure, and analyze music big data. \newline At Chartmetric, I worked on front-end engineering and data visualization.
  \end{Detail}


\BigGap
\Entry
  \href{https://ad.bigpearl.io}
  {\textbf{BigPearl}, Founding member, Front-end Engineering
  \hfill
  \DatestampYM{2017}{03} --
  \DatestampYM{2017}{07}
  }
  \begin{Detail}
  \SubBulletItem
    BigPearl is a MarTech startup that support advertisers to search for influencers and run ad campaigns. \newline At BigPearl, I participated in both business planning and front-end engineering.
  \end{Detail}


\Section
{Club\newline Activities}
{Club\newline Activities}
{PDF:ClubActivity}

\BigGap
\Entry
  \href{https://likelion.net}
  {\textbf{Likelion}, Web Programming Club, Seoul National University
  \hfill
  \DatestampYM{2016}{03} --
  \DatestampYM{2018}{06}
  }

  \begin{Detail}
  \SubBulletItem
    In Likelion, I tutored peer students on basic web programming.
  \end{Detail}

\newpage
    
\Section
{Other\newline Projects}
{Other\newline Projects}
{PDF:OtherProjects}
  
\BigGap
\Entry
  \href{https://github.com/kixlab/suggestbot_rails/}
  {\textbf{SuggestBot}, {\small{Kixlab project}}
  \hfill
  \DatestampYM{2018}{12} --
  \DatestampYM{2019}{02}
  }

  \begin{Detail}
  \SubBulletItem
    A crowdsourcing interface, developed with Ruby on Rails, that collects evidence of a piece of online discussion from IAC dataset.
  \end{Detail}

\BigGap
\Entry
  \href{https://github.com/jyoonsong/RAVi}
  {\textbf{Real-time Annotation for Video Chat through Collaborative Tagging}
  \hfill
  \DatestampYM{2018}{12} --
  \DatestampYM{2019}{02}
  }

  \begin{Detail}
  \SubBulletItem
    Inspired by Tilda (Zhang et al., 2018), this project designed a system that summarizes a video chat in real-time using the annotations collaboratively created by the chat participants.
  \end{Detail}


\BigGap
\Entry
  \href{https://jaeyoon.io/dt4c}
  {\textbf{Six Degrees of Jaeyoon Song}, {\small{Individual assignment for `Design Thinking' class}}
  \hfill
  \DatestampYM{2018}{10} --
  \DatestampYM{2018}{11}
  }

  \begin{Detail}
  \SubBulletItem
    A visualization using d3.js in order to check whether the concept of `six degrees of Kevin Bacon' also applies to my relationships on Facebook.
  \end{Detail}

\BigGap
\Entry
  \href{https://jaeyoon.io/dt4c}
  {\textbf{Are Refugees Dangerous?}, {\small{Individual assignment for `Design Thinking' class}}
  \hfill
  \DatestampYM{2018}{10} --
  \DatestampYM{2018}{11}
  }

  \begin{Detail}
  \SubBulletItem
    An infographic indicating what people think, what data say, and what media highlight about refugees. 
  \end{Detail}

\BigGap
\Entry
  \href{https://jaeyoon.io/infovis}
  {\textbf{Korean Independence Movement}, {\small{Team assignment for `HCI' class}}
  \hfill
  \DatestampYM{2018}{05} --
  \DatestampYM{2018}{06}
  }
  \begin{Detail}
    \SubBulletItem
      A visualization of Korean Independence Movement created by d3.js.
  \end{Detail}

\BigGap
\Entry
  \href{https://jaeyoon.io}
  {\textbf{The Cube}, {\small{Personal project}}
  \hfill
  \DatestampYM{2017}{10} --
  \DatestampYM{2017}{11}
  }
  \begin{Detail}
    \SubBulletItem
      My previous portfolio website to work as a freelance UX engineer.
  \end{Detail}


\BigGap
\Entry
  \href{http://shashagungun.com}
  {\textbf{Shashagungun}, {\small{Personal project}}
  \hfill
  \DatestampYM{2016}{12} --
  \DatestampYM{2017}{02}
  }
  \begin{Detail}
    \SubBulletItem
      A web platform using Ruby on Rails that gathers posters of various school events.
  \end{Detail}


%%%%%%%%%%%%
%% SKILLS %%
%%%%%%%%%%%%
\BigGap
\Section
{Skills}
{Skills}
{PDF:Skills}

\SubSection
{Languages}
{Languages}
{PDF:Languages}

\BigGap
\BulletItem
Korean: Native proficiency.

\Gap
\BulletItem
English: Full professional proficiency.
\begin{Detail}
\SubBulletItem
GRE - Verbal 164 / Quantitative 170 / Writing 4.5 (Oct 2019)
\end{Detail}

\Gap
\BulletItem
Japanese: Intermediate (reading); basic (speaking, writing).
\begin{Detail}
\SubBulletItem
JLPT - N2 (Jan 2014)
\end{Detail}

\BigGap
\SubSection
{Programming}
{Programming}
{PDF:Programming}

\BigGap
\BulletItem
\textbf{JavaScript} (D3.js, React.js, TweenLite.js, jQuery, ...)

\Gap
\BulletItem
Ruby on Rails,
SASS/SCSS

\Gap
\BulletItem
Python, Java, C++

\BigGap
\SubSection
{Others}
{Others}
{PDF:Others}

\BigGap
\BulletItem
Sketch App,
Adobe Photoshop.

\Gap
\BulletItem
{\LaTeX},
Microsoft Word,
Microsoft Excel,
Microsoft PowerPoint.

\end{Body}

%%%%%%%%%%%
% CV NOTE %
%%%%%%%%%%%

\BigGap
\UseNoteFont%
\null\hfill%
[\textit{\CVNote}]

\end{document}
